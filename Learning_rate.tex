\documentclass[12pt]{article}\usepackage[]{graphicx}\usepackage[]{color}


\usepackage[utf8]{inputenc}
\usepackage[english]{babel}


\begin{document}

\section{Learning rate}

The learning rate $w$ governs the strength of the vector updates.
Considering the base $x$ of a vector from a short read aligned to a base $y$ in the reference sequence, at iteration $i$, updates are performed using the following equation
\begin{equation}
y_{i+1}=w*y_{i}+(1-w)*x
\end{equation}


\subsection{Selecting learning rate}
The average coverage is calculated to choose the value of the learning rate as it will determine the expected number of iteration.
Considering a value of $y=0$, we want the rate strong enough to bring it to $1$ so that substitution are fully defined at the end of the learning.

In practice, we use half the average coverage to cauilcate the learning rate so that variation in the coverage is taken into account.
To set the threshold for INDELS, half of the learning rate is used.


\subsection{Issue}
The order with which the reads are chosen will have an impact on the update of the vector values.
Therefore, alternative caulcation for the update, e.g. weighted iterative mean, could be utilised.

\end{document}

